\section{Curve25519 Key Exchange}\label{curve25519-key-exchange}

\emph{Author of section: Andres Erbsen}

As classical Diffie-Hellman-Merkle key-exchange requires hundreds of
modular arithmetic operations on multiple-thousand-bit numbers to be
secure, we will be using a modern variation where every modular
multiplication is replaced with the addition of two points on a
carefully chosen elliptic curve{[}@Curve25519{]}. The other relevant
properties of elliptic curve addition are the same as for modular
multiplication, so we will continue to use the classical notation. Even
though one elliptic curve addition uses 10 modular multiplications, the
same level of security can be achieved using numbers that are 10 times
shorter. Numbers that are 10 times shorter are 100 times easier to
multiply, leading to 10 times less chip area and time usage.

The elliptic curve cryptography implementation is the most technically
involved part of this project. {[}@Curve25519{]} gives explicit formulas
for the elliptic curve arithmetic in terms of operations on integers
modulo the prime \(p=2^{255} -19\). Our implementation of the
\texttt{curve25519} module follows the figure presented in the appendix
of that paper and makes use of the properties of our modular arithmetic
modules to provide better performance. The main computation consists of
255 iterations of the elliptic curve Montgomery ladder step operation,
each of which involves 10 modular multiplications and a couple of
additions and subtractions. To allow for a simpler and faster
implementation, the intermediate results are stored as fractions and the
final output fraction is reduced to a scalar at the very end, requiring
modular division. To save area, the circuit has only one copy of the
modular multiplication unit and one add/subtract unit; these are used in
sequence to compute the elliptic curve operation and the division.
Furthermore, as the latency of our modular multiplication unit is twice
as high as the latency of the addition/subtraction unit, but the
throughput is the same, we did our best to keep the multiplication
pipeline active at all times. This results in using 7 255-bit registers
to store the intermediate values, in addition to the internal registers
of the modular arithmetic units. All in all, our implementation requires
less than 70000 cycles to perform an elliptic curve operation (public
key generation or shared key generation). Our implementation is a
trade-off between speed, circuit area, and complexity. We believe that
more careful pipeline management could offer slightly better speeds for
any area, storing the intermediate values of the elliptic curve
operations in block RAM instead of registers would allow for a smaller
area at the expense of speed, and implementing a dedicated modular
division (inversion) unit would allow for significantly better
performance at even larger expense of area.

\subsection{Modular multiplication}\label{modular-multiplication}

We took advantage of the fact that the modulus (\(p=2^{255}-19\)) was
known at the design time, that it is very close to a power of two, and
the availability of 18-by-18-bit multipliers to create implement an
efficient modular multiplication unit. The overall strategy goes as
follows:

\begin{enumerate}
\def\labelenumi{\arabic{enumi}.}
\itemsep1pt\parskip0pt\parsep0pt
\item
  Interpret each 255-bit input as 15 17-bit digits. While it would have
  been possible to use 18-bit digits, choosing 17 greatly simplifies the
  implementation because 255 bits can be evenly divided into 17-bit
  digits but not into 18-bit digits.
\item
  Perform an algorithm similar to schoolbook multiplication, where

  \begin{itemize}
  \itemsep1pt\parskip0pt\parsep0pt
  \item
    During each clock cycle, one row of partial products is computed
  \item
    Each partial product that would eventually overflow the 255-bit
    result because of its position is omitted from the calculation.
  \item
    However, the overflowing partial product is not discarded. As the
    modulus is \emph{not} a power of two, correct for the difference
    between two's complement integer overflow and addition mod \(p\) by
    adding 19 times the number of times the overflow wrapped around to
    the result. Because there must be an empty low order partial product
    slot for each partial product that is statically known to overflow,
    no addition needs to be performed: the just system places 19 times
    the overflowing partial product to the correct slot.
  \end{itemize}
\item
  Cumulatively add up the columns of partial products, but do not handle
  carries between them. The sum of each column has an upper bound of 42
  bits because it is a sum of 17 products of two 17-bit numbers times
  19.
\item
  Handle carries from the two most significant columns, adding the
  number of overflows times 19 to the result as before.
\item
  Handle all remaining carries starting from the least significant
  column and moving towards towards the most significant column. The
  result will be between \(0\) and \(2p\).
\item
  If the result overflows (has a carry of 1), subtract \(p\) from it.
  This is \emph{not} implemented as a separate step. Instead, there are
  two copies of steps 4 and 5, one of which works as described and the
  other subtracts the appropriate digit of \(p\) while handling each
  carry. The correct output is selected amongst the two branches using
  the carry bit.
\end{enumerate}

The breakdown of time usage is roughly as follows: step 2 takes 17
cycles, step 4 takes 1 cycle, and step 5 takes 17 cycles. As the FPGA
provides fast 17-by-17-bit multipliers in dedicated silicon, the main
area usage is due to the the accumulator and operand registers, and the
42-bit adders used for adding up columns and propagating carries.

\subsection{Modular Addition and
Subtraction}\label{modular-addition-and-subtraction}

Addition and subtraction also operate on 15 17-bit digits. While a
larger digit size would have offered superior speed at the submodule
level, we chose to stay consistent with the multiplier implementation
because the addition and subtraction latency is currently not the
bottleneck in the overall system. The general algorithm closely follows
the schoolbook method, and can also be seen as a ripple-carry adder
where a 1-by-1-bit ``full adder'' is replaced with a 17-by-17-bit digit
adder. As in multiplication, we need to ensure that the result of the
operation wraps modulo \(p=2^{255}-19\). Unlike multiplication, the
carry/borrow can only be a single bit, so there is no need for special
handling of carries from higher digits. This allows us to implement a
modular addition (resp. subtraction) of the inputs \(a\) and \(b\) by
first computing both \(a+b\) and \(a+b-p\) (resp. \(a-b\) and \(a-b+p\))
and selecting the one which is in the valid range to output when the
final carry/borrow bit becomes available.

Currently our system has separate circuitry for addition and
subtraction, but only one of them is used at once. We believe that it
would be possible to save some circuit area at negligible speed cost by
having one circuit that allows the operation to be indicated using an
input.

\subsection{Modular Inversion
(Division)}\label{modular-inversion-division}

Our system computes \(\frac{b}{a}\) as \(b\cdot\frac{1}{a}\).
Calculating \(\frac{1}{a}\) is implemented through exponentiation and
multiplication by Fermat's little theorem (\(a^{p-2} = a^{-1}\) mod p),
and exponentiation is done as repeated squaring and multiplication:
\(x^n= x \, ( x^{2})^{\frac{n - 1}{2}} \mbox{ if } n \mbox{ is odd, otherwise } (x^{2})^{\frac{n}{2}}\).
While we are aware of more complicated methods that allow to perform
modular inversion faster, we chose this one because it requires almost
no additional circuit area. Currently, modular inversion accounts for
one fifth of the total modular multiplications and one fourth of the
running time (because the ladder step is pipelined and inversion is
not).

\section{ChaCha20 Stream Cipher}\label{chacha20-stream-cipher}

ChaCha20 is a modern stream cipher. Given a secret key, it provides fast
random access to different positions of \(2^{130}\)-byte keystream which
is XOR-ed with the data to encrypt it. The procedure to compute a
64-byte block of keystream consists of 20 rounds, each of which mutates
a 4-by-4 table of 32-bit words by applying the \emph{quarter round}
function to all columns or all diagonals. A quarter round takes 4 32-bit
inputs and produces 4 32-bit outputs, mixing the bits of the input using
addition, rotation and XOR. Our circuit has four instantiations of the
quarter round module, and one of the twenty rounds is completed each
clock cycle. Finally, the output is computed by adding the initial table
entries to the final table entries (round 21 in our implementation).

This implementation provides good performance (3 bytes/cycle), but the
propagation delay of the circuit is rather large, limiting us to clock
frequencies under 50MHz. In our case it was not an issue (we are using a
27MHz), but a higher frequency implementation would probably need to
pipeline the quarter round function.

\section{Keccak hash function}\label{keccak-hash-function}

We used an open-source Keccak implementation by Homer Hsing, available
at \texttt{opencores.org}. The input is sent to the Keccak module in
chunks of 1 to 4 bytes, the output is a 512-bit high-quality
pseudo-random blob generated as a function of the input. The
implementation we used advertises a speed of 2.4Gbit/s at 100MHz, but we
used it at a much slower rate (less than 1Mbit/s).

\documentclass[a4paper]{report}

\usepackage[english]{babel}
\usepackage[utf8]{inputenc}
\usepackage{amsmath}
\usepackage{graphicx}
\usepackage[colorinlistoftodos]{todonotes}

\title{A Cryptographically Protected Phone System}

\author{Andres Erbsen and Adam Yedidia}

\date{\today}

\begin{document}
\maketitle

\section{Introduction}

In today's excessively monitored and public world, there is no more privacy. An
email can be read by Google, and a chat conversation can be read by Facebook. A
conversation over a cell phone can be intercepted at a cell phone tower, and a
phone conversation over a landline can be wiretapped. In the era of technology,
there is no longer a way to communicate privately except face-to-face.

No way, that is, except for cryptographically protected communications. The last
century, in addition to bringing about the technologies mentioned above for
communicating remotely (along with the technologies used to eavesdrop on them)
also brought about the invention of clever cryptographic schemes for hiding
information securely. These schemes go by many names; in our implementation, we
made use of a particular implementation the one called Diffie-Hellman key
exchange. The truly incredible thing about Diffie-Hellman key exchange (along
with a handful of other schemes which we will not use) is that it enables two
interlocutors \emph{with no prior agreements or contact} to communicate securely
in the presence of an eavesdropper \emph{who can hear everything they ever say
to each other}. Conditioned on the hardness of various problems which are widely
believed to in fact be very difficult to solve by the wider academic community,
it is in fact provably intractable for the eavesdropper to hear the
communications.

In what follows, we will give a detailed description a hardware implementation of a telephone
system that will be cryptographically protected in a theoretically defensible
way; that is, for an attacker to be able to eavesdrop on a call using our
telephone system would require either a long-standing conjecture to be false, or
would require more vastly more computational power than is available to the
world today. We begin this report with a high-level description of the ideas and concepts behind what we did; in the second half of the 

\chapter{High-Level Description}

\section{Transmitting Audio}

In order to transmit the audio across the channel, we will organize the audio
signal into packets, to be sent one at a time across the channel. It is more
efficient to perform the cryptographic calculations when the input is in
available in larger chunks than a single audio sample. Moreover, each packet
requires a header, which will contain a 128-bit cryptographic authenticator and
some minimal information. The first 8 bits will indicate whether the packet
is a data packet or some other specialized signal to the system, such as a
command or a preliminary communication. The next 16 bits will contain a sequence
number unique to each packet. The next 24 bits will contain a binary number that
holds the number of bytes held in the remainder of the packet.  Finally, the
rest of the packet holds the encrypted audio data, followed by the 128-bit
authenticator. The header is included in the calculation of the authenticator
and the length of the authenticator is included in the 24-bit length field.

Thanks to this flexible design which allows us to vary the number of data bits
in a packet, we will be able to tweak the number of bits we send at once because
it is not immediately obvious what the value should be; a smaller value will
increase the relative overhead of the packet header and make the cryptographic
calculations less efficient. A larger packet size, however, will increase the
latency of the system, because the system will have to accumulate more bits
before it can send a packet. We are likely to want a packet that is as small as
possible without further complicating the implementation.

In order to assemble a packet, we will have a buffer module which will
accumulate the bits before sending them all at once in packet form. The buffer
will make use of the BRAMs in order to be as efficient as possible, and because
it will be storing a potentially large amount of memory (so simply using
flip-flops would be a waste of valuable programmable logic).

Finally, the packets themselves will be transmitted over a simple wire that
connects two FPGA labkits, after having been encrypted by the encryption module.
We hope that this simple method will suffice for sending a sufficiently clear
signal; however, it is possible that the length of the wire necessary to
transmit from one labkit to another will produce artifacts, so if that turns out
to be the case we will ``clean'' the signal with a MAX395 chip, a chip designed
to drive signals cleanly across long distances.

\section{Cryptographic protection}

As it is easy for an individual to come up with a cryptosystem that they
themselves cannot break, but harder to create one that will withstand attacks
from others, we will be basing our design on existing well-audited building
blocks. As we are not aware of any freely available Verilog implementations of
these algorithms, we will be implementing them and verify their correctness by
comparing the behavior of our hardware implementations to the existing reference
(software) implementations.

The overall plan is as follows: in the very beginning of each call, the two
communicating labkits will perform a Diffie-Hellman handshake to generate a
shared secret. That secret will be used to seed a pseudo-random keystream
generator and the digital audio signal will be xor-ed with the keystream.
This prevents the attacker (who does not know the shared secred or the
keystream) from recovering the audio signal. To also prevent the attacker from
selectively modifying the audio, each audio packet will be accompanied by a
128-bit authenticator that can only be computed for a message by a party that
knows the secret key -- the callers, but not the attacker.

\begin{description}
  \item[Call Initialization:] Participant A chooses random $a$ and reveals $g^a$,
	  B chooses $b$, reveals $g^b$; both compute $s=g^{ab}$. $g$ is a public constant.
  \item[Encryption:] $\text{encrypted}_i = \text{data}_i \mathbin{\oplus} \text{keystream}(s,i)$
  \item[Decryption:] $\text{data}_i = \text{encrypted}_i \mathbin{\oplus} \text{keystream}(s,i)$
  \item[Packetization:] $\text{packet}_i = \left(type, i, \text{length}(\text{encrypted}_i), \text{encrypted}_i\right)$
  \item[Authentication:] $\text{authenticator}_i = \text{hash}(s, \text{packet}_i)$
  \item[Auth. checking:] Accept received packet iff $\text{hash}(s, \text{packet}_i) = \text{authenticator}_i$.
\end{description}

The described mechanisms are sufficient against an attacker that can only see
what is sent over the wire but tamper with it; for example, one might cut the
wire and connect it to their phone instead and all this would continue without
interruption. Furthermore, they might make another call to the intended
recipient and connect these calls while maintaining the ability to eavesdrop. To
make sure that the two users of ours system are indeed on the same call (and
thus not eavesdropped on), we display a value derived from the handshake inputs
($g^a$ and $g^b$) to both of them -- if it matches, the connection hasn't been
tampered with. Displaying just $g^a$ and $g^b$ would also work, except that each
of them is 256 bits long and comparing them would be tedious. Instead, we
display the first (labkit's hex display width?) bits of $\text{hash}(g^a, g^b)$,
where the hash function is such that changing just one of the inputs would
chaotically change the output.

As classical Diffie-Hellman requires hundreds of modular arithmetic operations
on multiple-thousand-bit numbers to be secure, we will be using a modern
variation where every modular multiplication is replaced with the addition of
two points on a carefully chosen elliptic curve. The other relevant properties
of elliptic curve addition are the same as for modular multiplication, we will
even continue to use the classical notation. Even though one elliptic curve
addition uses more than one modular multiplication, smaller numbers can be used
without compromising security. Our choice of primitives goes as follows:

\begin{description}
  \item[Random numbers:] Hash of 100ms of microphone input.
  \item[Diffie-Hellman function:] Curve25519 (uses arithmetic modulo $2^{255}-19$)
  \item[Keystream:] ChaCha20 (uses 32-bit addition, xor, and rotation by constant).
  \item[Hash:] BLAKE (uses 32-bit addition, xor, and rotation by constant).
\end{description}

The last two are fairly straightforward to implement given the specification, as
is implementing elliptic curve Diffie-Hellman if the underlying modular
arithmetic is given. Arithmetic modulo $2^{255}-19$ is a little bit more
tricky. Addition comes down to first adding the inputs mod $2^{255}$ and then
subtracting $2^{255}-19$ if the result overflowed (both branches can be computed
in parallel); subtraction is similar.  Multiplication is performed by breaking
the 255-bit number into 15 17-bit digits and applying the schoolbook
multiplication algorithm, one row of partial products at a time. To modify the
schoolbook multiplication to work modulo $2^{255}$, the partial products that
would be too far on the higher order side are wrap over back to the lower order
digits; the difference between working modulo $2^{255}$ and module $2^{255}-19$
can be corrected for by multiplying all wrapped-around partial products by 19.
The partial products for each column are added up as they are generated. To
compute the final 255-bit answer, we first handle carries from the higher-order
digits that definitely cause the final answer to pass $2^{255}-19$ by adding the
number of passes times 19 to the lowest order digit. As the carries from the
lower order digits to the higher order digits might also cause the answer to
pass $2^{255}-19$ (but only once), answer$-2^{255}-19$ is computed together with
the final answer when the last carries are handled and the one that is in the
valid range is returned. Exponentiation is implemented using the standard
square-and-multiply technique, Fermat's little theorem is used to provide a slow
but simple division circuit.

\section{Resulting User Experience}

Two labkits are connected with a couple of wires. When a connection is
established, both hex displays show the same value. Anything spoken into one
labkit's microphone is heard from the headphones of the other and vice versa.

\section{Possible Attacks}

It is the nature of practical cryptography that it is very difficult to prove the security of your system. In order to prove that nobody can break your encryption in any way, it would be necessary to prove something about the entire space of possible attacks, which is too great to comprehend, or it would be necessary to enumerate eve1y possible attack to the system and show that each fails, which is impossible. Unfortunately, the best we can do at this point is enumerate all widely-known attacks against cryptographic systems, and explain why each one cannot succeed against our system. It is possible (even likely) that there exists some attack not described here that would break our system; unfortunately, the best measure of security that we can offer against such an attack is that because people as a whole haven't thought of it yet, it's likely that a potential attacker wouldn't either.

\subsection{Brute-Force Attacks} 

This class of possible attacks has as its goal to learn through algorithmic means either of the interlocutors' secret keys, which will be enough to let the attacker compute the shared secret and the decryption of the message. Assuming that division in the space of elliptic curve functions is computationally difficult (which is widely believed in the academic community to be true) this class of possible attacks won't succeed against our system because of the fact that the only algorithms able to find an interlocutor's secret key as a function of their public key take exponential time; moreover, our key length is long enough that no implementation of any algorithm has come close to being able to do this.

\subsection{Man-in-the-Middle Attacks}

This type of attack is based off the idea of spoofing the two interlocutors into thinking that they are talking to each other, when in fact, they are both talking to the attacker and the attacker is relaying messages between them. No amount of security on the channel itself can protect against an attack of this kind, since the attacker's ability to hear the communications comes from the fact that two participants in a phone conversation can hear the communications--something that is necessary for the basic functioning of a telephone system. However, we can give t the interlocutors the ability to know when there is a man-in-the-middle attacker, by giving everyone access to the public key of a given other person, and also giving them access to the public key of the person they're talking to. If the Alice believes herself to be speaking to Bob, but is actually speaking to Eve, who is executing a man-in-the-middle attack, then she will see both Bob and Eve's public keys displayed on her phone's screen, and this will tell her that she is not speaking to the person she think she is. This is the basic idea; in reality, we would display not the public key itself (which is much too long to display) but a hash thereof, and if Bob and Eve's public keys were different, then the hashes would be virtually certain to also be different, assuming we used a cryptographic hash function (a hash function which is computationally difficult to invert).

\subsection{Data Manipulation Attacks}

These attacks consist of manipulating the data cleverly; for example, it might be imagined that in an insecure cryptographic system, an attacker might be able to computer the \verb|XOR| of the data on the wire from Alice to Bob and the wire from Bob to Alice, for example. % I will write more in a bit

\chapter{Detailed Implementation}

In this section, we give a detailed description of our design and how it works. In the pages that follow, we provide two block diagrams that provide a visual summary of our design. In the text, we describe in detail the working of each of the modules shown in the block diagram.

\begin{figure}
\label{fig3}
\centering \caption{A block diagram of the execution of Phase 1. Thick lines indicate external wires (which the attacker has access to), whereas arrows indicate internal wires (which the attacker does not have access to). For clarity, we show only the information flowing from Alice to Bob; in reality, this entire design is duplicated in the other direction.}
\makebox[\textwidth][c]{\includegraphics[width=1.7\textwidth]{figs/phase1.png}}%
\end{figure}

\begin{figure}
\label{fig3}
\centering \caption{A block diagram of the execution of Phase 2. Thick lines indicate external wires (which the attacker has access to), whereas arrows indicate internal wires (which the attacker does not have access to). For clarity, we show only the information flowing from Alice to Bob; in reality, this entire design is duplicated in the other direction.}
\makebox[\textwidth][c]{\includegraphics[width=1.7\textwidth]{figs/phase2.png}}%
\end{figure}

\section{Controller}

This part of the code did not have its own module, nor did it consist of very much distinct code; however, it is present in the block diagram as a design abstraction, which controls the flow of the system from Phase 1 (handshake) into Phase 2 (audio). Upon turning on the labkit, it tells the Handshake FSM module to begin the handshake; upon successful completion of the handshake, it takes the shared key output by the Handshake FSM module, feeds it into the Chacha20 module, and begins Phase 2. It can be thought of as a very high-level state machine with two states over the entire design.

\section{Handshake FSM}

This module is a finite-state machine in charge of making sure of the proper execution of the Diffie-Hellman handshake that will communicate the shared key to both participants. It consists of the following seven states, each one leading directly into the next: \\ \\
1. Waiting for start pulse: the state machine is idle and waiting to be told to begin. \\
2. Computing the public key: the state machine takes as input a secret key seed, feeds it into the Curve25519 module, and waits for the computation to terminate. \\
3. Sending the public key (no acknowledgement): the state machine sends its public key over the wire in a handshake packet, with an acknowledgement byte indicating that it has not yet received a handshake packet from its partner. It waits to receive a handshake packet from its partner. \\
4. Sending the public key (with acknowledgement): the state machine sends its public key over the in a handshake packet, with an acknowledgement byte indicating that it has successfully received a handshake packet from its partner. It waits to receive a handshake packet containing an acknowledgement byte confirming successful receipt in the other direction. \\
5. Reading the partner's public key: at this point, both partners have confirmed that the other has succeeded in receiving a handshake packet. The state machine proceeds to read its partner's public key. 
6. Computing the shared key: the state machine generates the shared key by feeding both its secret key seed and its partner's public key and feeding them into the Curve25519 module. \\
7. Handshake complete: the state machine is idle and outputs the shared key. \\

\section{Serial Port Modules}

This pair of modules, SendFrame and ReceiveFrame, are in charge of sending data over the wire that connects the two labkits. SendFrame sends a byte of data over the wire by repeatedly calling SendBit on both the packet header and whatever information follows. \\

SendBit sends a single bit over the wire, by first setting the data wire to whatever that voltage that bit corresponds to, and then setting the \verb|DATA READY| wire to high one clock cycle later (indicating that data is available to be read on the data line). \\

ReceiveFrame reads a single byte off the data wire by calling ReadBit on the data wire whenever the \verb|DATA READY| signal goes high. ReadBit reads the data off the wire. ReceiveFrame knows to start parsing the information on the wire as relevant as soon as a packet header is seen. \\

Because of the fact that our roughly 500mm-long wire is imperfect, errors in communication routinely occur over the wire if each bit is read off the wire only once (about one to ten errors per second). These errors are still infrequent compared to the frequency with which sound data is sent (one byte per 20 $\mu s$), but these errors can produce unpleasant popping noises over the phone line, and worse, if such an error is made during the Diffie-Hellman handshake, one of the two partners may get the wrong public key data, causing the two partners to differ about the value of the shared key and rendering all communication impossible. \\

For this reason, the ReceiveBit module ``oversamples'' the data line, meaning that it samples three times, and takes the bit it saw most frequently from those three samples. Because errors happen roughly once in a million bits sent over the wire, the probability that two out of three given samples will be erroneous (making the possibly-dubious assumption that each sample is independent) is roughly one in 300 billion. This vastly reduced probability of error was enough for us to never observe any popping over the wire.

\section{Packet Structure}

Our system makes use of two different types of data packets that can be sent over the wire. We call the two types of packets \emph{handshake packets} and \emph{audio packets} respectively.  \\

% handshake packet diagram goes here

% caption
Both packets begin with a \emph{packet header}: a string of bits, 10101011, that gets sent over the wire to notify the receiving labkit that transmission of the packet body is about to begin.  \\

The packet body is different for each packet. Handshake packets contain the following components, which make up 272 bits: \\ \\
-An acknowledgement byte (8 bits), depending on whether or not the labkit sending the handshake packet has yet successfully received a handshake packet from its partner. An acknowledgement byte of 01100110 means ``I have not yet received a handshake packet from you'' and an acknowledgement byte of 100110011 means ``I have received a handshake packet from you.''  \\
-The transmitter's unmodified public key (256 bits). Note that because this key is being transmitted over the channel without any modification, an attacker can read the public key; however, due to modern cryptographic hardness conjectures, an attacker with access to both public keys would not be able to find the shared key without running an algorithm that would take years if not millenia to complete (without drastically improving state-of-the-art elliptic-curve-discrete-logarithm algorithms).

Audio packets contain the following components, which make up 552 bits: \\ \\
-A sequence number (32 bits). If this is the $i^{\textrm{th}}$ packet that has been sent over the wire during this phone conversation, then the sequence number would have a value of $i$. Note that because our sequence numbers have a fixed length of 32 bits, no more than $2^{32} \approx $ 4 billion packets can be sent over the wire before the conversation becomes insecure; however, given that we only send 48,000 packets per second over the wire, the phone conversation can last about 100,000 seconds or roughly one day before it is insecure: more than the length of most any phone conversation between two ordinary people. If this design was to be used in a situation where communication was going to last for weeks or months, it would be wise to increase the size of this sequence number to 64 bits. \\
-Encrypted audio data (512 bits). This is the result of a bitwise XOR operation that has been performed between the audio data and the output of the Chacha20 module, which outputs the cryptographic bit-stream. An attacker with access to this audio data and the public keys of both participants in the conversation cannot in a practical way deduce what is being said in the conversation. \\

\begin{figure}
\label{fig1}
\centering \caption{The layout of the two types of packets.} \includegraphics[width=\textwidth]{figs/packets.png}
\end{figure}

\section{Buffer Logic}

In the code, we did not give this part of the program an actual separate module, but instead put it directly into \verb|labkit.v|. However, both the few lines contained in the execution of the logic and our behavior in how we structured it belie its complexity; this was a very difficult module, both to think about and to execute properly. \\

The idea of the module is that it must fill buffers with data (be it audio data or handshake data), so that it can be passed, preceded by a packet header, to the serial-port code, which can then send it over the wire. Unfortunately, this is complicated by the fact that if a buffer were to be modified during sending, that could lead to malformed packets being sent: a potential irritant for the user if it happens while audio data is being sent, and potentially catastrophic if it happens during the Diffie-Hellman handshake. \\

To avoid this possibility, the module uses two BRAMs acting as buffers on both the sending and the receiving side, each with 64 slots containing one byte each. While the data from one of the two buffers is being sent over the wire, the other buffer is free to be written into for the next send. Then, when the send on the current buffer finishes, the two buffers switch roles, with the first being ready to be written into and the second relaying its data to the serial port logic to be sent.

\section{Curve25519}

The elliptic curve cryptography implementation is the most technically involved
part of this project. [@Curve25519] gives explicit formulas for the elliptic
curve arithmetic in terms of operations on integers modulo $2^{255} - 19$. Our
implementation of the `curve25519` module follows the figure presented in the
appendix of that paper and makes use of the properties of our modular arithmetic
modules to provide better performance. To save area, the circuit has only one
copy of the modular multiplication unit and one add/subtract unit; these are
used in sequence to compute the elliptic curve operation. Furthermore, as the
latency of our modular multiplication unit is twice as high as the latency of
the addition/subtraction unit, but the throughput is the same, we did our best
to keep the multiplication pipeline active at all times. This results in using 7
255-bit registers to store the intermediate values, in addition to the internal
registers of the modular arithmetic units. All in all, our implementation
requires less than 70000 cycles to perform an elliptic curve operation (public
key generation or shared key generation). Our implementation is a trade-off
between speed, circuit area, and complexity. We believe that more careful
pipeline management could offer slightly better speeds for any area, storing the
intermediate values of the elliptic curve operations in block RAM instead of
registers would allow for a smaller area at the expense of speed, and
implementing a dedicated modular division (inversion) unit would allow for
significantly better performance at even larger expense of area.

\section{Modular Multiplication}

We took advantage of the pseudo-Mersenne structure of the modulus
($p=2^{255}-19$) and the availability of 18-by-18-bit multipliers to create
implement an efficient modular multiplication unit. The overall strategy goes as
follows: \\ \\
1. Interpret each 255-bit input as 15 17-bit digits. While it would have been
possible to use 18-bit digits, choosing 17 greatly simplifies the implementation
because 255 bits can be evenly divided into 17-bit digits but not into 18-bit
digits. \\
2. Perform an algorithm similar to schoolbook multiplication, where \\
- During each clock cycle, one row of partial products is computed \\
- Each partial product that would eventually overflow the 255-bit result
because of its position is omitted from the calculation. \\
- However, the overflowing partial product is not discarded. As the modulus
is \emph{not} a power of two, correct for the difference between two's
complement integer overflow and addition mod $p$ by adding 19 times the number
of times the overflow wrapped around to the result. Because there must be an
empty low order partial product slot for each partial product that is statically
known to overflow, no addition needs to be performed: the just system places 19
times the overflowing partial product to the correct slot. \\
3. Cumulatively add up the columns of partial products, but do not handle
carries between them. The sum of each column has an upper bound of 42 bits
because it is a sum of 17 products of two 17-bit numbers times 19. \\
4. Handle carries from the two most significant columns, adding the number of
overflows times 19 to the result as before. \\
5. Handle all remaining carries starting from the least significant column and
moving towards towards the most significant column. The result will be
between $0$ and $2p$. \\
6. If the result overflows (has a carry of 1), subtract $p$ from it. This is \emph{not}  implemented as a
separate step. Instead, there are two copies of steps 4 and 5, one of which
works as described and the other subtracts the appropriate digit of $p$ while
handling each carry. The correct output is selected amongst the two branches
using the carry bit. \\

The breakdown of time usage is roughly as follows: step 2 takes 17 cycles, step
4 takes 1 cycle, and step 5 takes 17 cycles. As the FPGA provides fast
17-by-17-bit multipliers in dedicated silicon, the main area usage is due to the
the accumulator registers and the 42-bit adders used for adding up columns and
propagating carries.

\end{document}
